\subsection{Problem Set 3 Code}

\textbf{sim\_config.m}
\begin{lstlisting}
% Input initial conditions in orbital elements
% Case 1: Circular orbit
initial_conditions_chief_circular = [6780000, 0.001, 0, 0, 0,.1]; 
initial_conditions_deputy_circular = [0; 10; 0; 1; 0; 0];

% Case 2: Eccentric orbit
initial_conditions_chief_eccentric = [6780000, 0.1, 0.1, 0, 0, 0]; 
initial_conditions_deputy_eccentric = [0; 10; 0; 1; 0; 0];

% Case 3: Semi-major axis difference
initial_conditions_chief_sma = [6780000, 0.1, 0.1, 0, 0, 0]; 
initial_conditions_deputy_sma = [10; 0; 0; 0; 0; 0];  % 10m radial offset

% Case 4: High eccentricity
initial_conditions_chief_high_ecc = [6780000, 0.6, 0.1, 0, 0, 0]; 
initial_conditions_deputy_high_ecc = [0; 10; 0; 1; 0; 0];

% Time step and number of orbits
num_orbits = 15;
time_step = 1;

% Common simulation settings
simulation_settings = struct();
simulation_settings.numerical_propogation = true;
simulation_settings.keplerian_propogation = false;
simulation_settings.J2 = false;
simulation_settings.relative_deputy = false;
simulation_settings.absolute_deputy = true;
simulation_settings.hcw_deputy = false;
simulation_settings.ya_deputy = true;
simulation_settings.roe_circular_deputy = false;
simulation_settings.roe_eccentric_deputy = true;
simulation_settings.create_bounded_motion = false;

% Common graphics settings
graphics_settings = struct();
graphics_settings.orbit_eci = false;
graphics_settings.compare_numerical_vs_kepler = false;
graphics_settings.plot_orbital_elements.base_elems = false;
graphics_settings.plot_orbital_elements.J2_mean_analytical_solution = false;
graphics_settings.plot_eccentricity_vector = false;
graphics_settings.plot_ang_momentum_vector = false;
graphics_settings.plot_specific_energy = false;
graphics_settings.plot_nonlinear_comparison = false;
graphics_settings.plot_propagation_errors = false;
graphics_settings.plot_high_eccentricity = false;

graphics_settings.plot_deputy = struct();
graphics_settings.plot_deputy.relative = false;
graphics_settings.plot_deputy.absolute = true;
graphics_settings.plot_deputy.hcw = false;
graphics_settings.plot_deputy.ya = true;
graphics_settings.plot_deputy.roe_circular = false;
graphics_settings.plot_deputy.roe_eccentric = true;
graphics_settings.plot_deputy.manuvered = false;

% Case 1: HCW Equations (Circular Orbit)
fprintf('Running HCW equations for near-circular orbit case\n');
simulation_settings.hcw_deputy = true;
simulation_settings.ya_deputy = false;
simulation_settings.roe_eccentric_deputy = false;

graphics_settings.plot_deputy.hcw = true;
graphics_settings.plot_deputy.ya = false;
graphics_settings.plot_deputy.roe_eccentric = false;

sim1 = simulator(initial_conditions_chief_circular, initial_conditions_deputy_circular, num_orbits, time_step, simulation_settings, graphics_settings);
sim1.run_simulator();

% Case 2: YA and ROE Equations (Eccentric Orbit)
fprintf('Running YA and ROE for eccentric orbit case\n');
simulation_settings.hcw_deputy = false;
simulation_settings.ya_deputy = true;
simulation_settings.roe_eccentric_deputy = true;

graphics_settings.plot_deputy.hcw = false;
graphics_settings.plot_deputy.ya = true;
graphics_settings.plot_deputy.roe_eccentric = true;

sim2 = simulator(initial_conditions_chief_eccentric, initial_conditions_deputy_eccentric, num_orbits, time_step, simulation_settings, graphics_settings);
sim2.run_simulator();

% Case 3: Nonlinear Comparison
fprintf('Running nonlinear comparison\n');
simulation_settings.absolute_deputy = true;
graphics_settings.plot_nonlinear_comparison = true;
graphics_settings.plot_propagation_errors = true;

sim3 = simulator(initial_conditions_chief_eccentric, initial_conditions_deputy_eccentric, 5, time_step, simulation_settings, graphics_settings);
sim3.run_simulator();

% Case 4: Semi-major axis difference
fprintf('Running SMA difference case\n');
graphics_settings.plot_nonlinear_comparison = false;
graphics_settings.plot_propagation_errors = false;

sim4 = simulator(initial_conditions_chief_sma, initial_conditions_deputy_sma, num_orbits, time_step, simulation_settings, graphics_settings);
result4 = sim4.run_simulator();

% Rename the figures directly to match the expected filenames
figList = findall(0, 'Type', 'figure');
for i = 1:length(figList)
    if strcmp(get(figList(i), 'Name'), 'RelativeTrajectory')
        saveas(figList(i), 'SMA_Difference_Position.png');
    elseif strcmp(get(figList(i), 'Name'), 'RelativeVelocity')
        saveas(figList(i), 'SMA_Difference_Velocity.png');
    end
end

% Case 5: High eccentricity
fprintf('Running high eccentricity case\n');
graphics_settings.plot_high_eccentricity = true;

sim5 = simulator(initial_conditions_chief_high_ecc, initial_conditions_deputy_high_ecc, 5, time_step, simulation_settings, graphics_settings);
sim5.run_simulator();
\end{lstlisting}

\textbf{dynamics.m (key functions)}
\begin{lstlisting}
function state_rtn = HCW_propogation(t, initial_conditions_chief_oes, initial_conditions_deputy_rtn)
    a = initial_conditions_chief_oes(1);
    n = sqrt(constants.mu / (a^3));
    a_matrix = [a, 0, 0, 0, 0, 0;
                0, a, 0, 0, 0, 0;
                0, 0, a, 0, 0, 0;
                0, 0, 0, a*n, 0, 0;
                0, 0, 0, 0, a*n, 0;
                0, 0, 0, 0, 0, a*n];

    integration_constants = inv(util.calculate_hcw_matrix(0, initial_conditions_chief_oes)) * inv(a_matrix) * initial_conditions_deputy_rtn;
    state_rtn = a_matrix * util.calculate_hcw_matrix(t, initial_conditions_chief_oes) * integration_constants;
end

function state_rtn = YA_propogation(f, t, initial_conditions_chief_oes, initial_conditions_deputy_rtn)
    a = initial_conditions_chief_oes(1);
    e = initial_conditions_chief_oes(2);
    n = sqrt(constants.mu / (a^3));
    nu = sqrt(1 - e^2);
    a_matrix = [a * nu^2, 0, 0, 0, 0, 0;
                0, a * nu^2, 0, 0, 0, 0;
                0, 0, a * nu^2, 0, 0, 0;
                0, 0, 0, a*n/nu, 0, 0;
                0, 0, 0, 0, a*n/nu, 0;
                0, 0, 0, 0, 0, a*n/nu];
    integration_constants = inv(util.calculate_ya_matrix(initial_conditions_chief_oes(6), 0, initial_conditions_chief_oes)) * inv(a_matrix) * initial_conditions_deputy_rtn;
    state_rtn = a_matrix * util.calculate_ya_matrix(f, t, initial_conditions_chief_oes) * integration_constants;
end

function state_rtn = propagate_with_roe_eccentric(t, initial_conditions_chief_oes, initial_roe, current_chief_oe)
    a = initial_conditions_chief_oes(1);
    e = initial_conditions_chief_oes(2);
    i = initial_conditions_chief_oes(3);

    da = initial_roe(1);
    dl = initial_roe(2);
    de_x = initial_roe(3);
    de_y = initial_roe(4);
    di_x = initial_roe(5);
    di_y = initial_roe(6);

    f = current_chief_oe(6);
    w = current_chief_oe(5);
    
    M = util.TtoM(f, e);
    M0 = util.TtoM(initial_conditions_chief_oes(6), e);
    
    n = sqrt(constants.mu/a^3);
    orbit_count = floor(n*t/(2*pi));
    dM = M - M0 + 2*pi*orbit_count;

    eta = sqrt(1 - e^2);
    k = 1 + e*cos(f);
    k_prime = -e*sin(f);

    e_x = e*cos(w);
    e_y = e*sin(w);

    dr_r = a * (da - (k*k_prime/eta^3)*dl - (de_x/eta^3)*k*cos(f) - (de_y/eta^3)*k*sin(f) + (k/eta^3)*((k-1)/(1+eta))*(e_x*de_x + e_y*de_y) + (k*k_prime/eta^3)*di_y*cot(i));
    dr_t = a * ((k^2/eta^3)*dl + (de_x/eta^2)*(1+k)*sin(f) - (de_y/eta^2)*(1+k)*cos(f) + (1/eta^3)*(eta + k^2/(1+eta))*(e_y*de_x - e_x*de_y) + (1 - k^2/eta^3)*di_y*cot(i)) - 1.5*a*da*dM;
    dr_n = a * (di_x*sin(f) - di_y*cos(f));

    dv_r = n * a * (eta/k) * (de_x*sin(f) - de_y*cos(f));
    dv_t = n * a * (eta/k) * (-1.5*da + 2*de_x*cos(f) + 2*de_y*sin(f));
    dv_n = n * a * (eta/k) * (di_x*cos(f) + di_y*sin(f));
    
    state_rtn = [dr_r; dr_t; dr_n; dv_r; dv_t; dv_n];
end
\end{lstlisting}

\textbf{util.m (key functions)}
\begin{lstlisting}
function ya_matrix = calculate_ya_matrix(f, t, oes)
    e = oes(2);
    a = oes(1);
    n = sqrt(constants.mu / (a^3));
    nu = sqrt(1 - e^2);
    k = 1 + (e * cos(f));
    kd = -e * sin(f);
    tau = n * t / (nu^3);

    ya_matrix = [(1/k) + (3*kd*tau/2), sin(f), cos(f), 0, 0, 0;
                -3*k*tau/2, (1 + (1/k))*cos(f), -(1 + (1/k))*sin(f), 1/k, 0, 0;
                0, 0, 0, 0, (1/k)*sin(f), (1/k)*cos(f);
                (kd/2)-(3*(k^2)*(k-1)*tau/2), (k^2)*cos(f), -(k^2)*sin(f), 0, 0, 0;
                -(3/2)*(k + (kd*tau*k^2)), -(k^2 + 1)*sin(f), -e-((1+k^2) * cos(f)), -kd, 0, 0;
                0, 0, 0, 0, e + cos(f), -sin(f)];
end

function roe = calculate_quasi_nonsingular_roe(chief_state, deputy_state)
    chief_oe = util.ECI2OE(chief_state);
    deputy_oe = util.ECI2OE(deputy_state);

    a_c = chief_oe(1);
    e_c = chief_oe(2);
    i_c = chief_oe(3);
    RAAN_c = chief_oe(4);
    w_c = chief_oe(5);
    nu_c = chief_oe(6);
    
    a_d = deputy_oe(1);
    e_d = deputy_oe(2);
    i_d = deputy_oe(3);
    RAAN_d = deputy_oe(4);
    w_d = deputy_oe(5);
    nu_d = deputy_oe(6);

    M_c = util.TtoM(nu_c, e_c);
    M_d = util.TtoM(nu_d, e_d);
    
    da = (a_d - a_c) / a_c;
    dl = (M_d + w_d) - (M_c + w_c) + (RAAN_d - RAAN_c)*cos(i_c);

    de_x = e_d*cos(w_d) - e_c*cos(w_c);
    de_y = e_d*sin(w_d) - e_c*sin(w_c);
    
    di_x = i_d - i_c;
    di_y = (RAAN_d - RAAN_c)*sin(i_c);
    
    roe = [da; dl; de_x; de_y; di_x; di_y];
end
\end{lstlisting}

\textbf{Modified plotter.m functions}
\begin{lstlisting}
function plot_nonlinear_comparison(result, graphics_settings)
    % Position figure
    fig_pos = figure('Name', 'Nonlinear_Position_Comparison');
    
    % Plot position components
    subplot(2,2,1);
    hold on;
    if isfield(result, 'absolute_state_history')
        plot(result.absolute_state_history(:,2), result.absolute_state_history(:,1), 'k-', 'LineWidth', 2);
    end
    if isfiel ```

function plot_high_eccentricity(result, graphics_settings)
    % Position figure
    fig_pos = figure('Name', 'High_Eccentricity_Position');
    
    subplot(2,2,1);
    hold on;
    legend_names = [];
    % Plot all deputies - code here similar to plot_deputy
    % But specifically for high eccentricity case
    
    % Save the figure
    saveas(fig_pos, 'High_Eccentricity_Position.png');
    
    % Velocity figure
    fig_vel = figure('Name', 'High_Eccentricity_Velocity');
    
    % Plot velocity components
    
    % Save the figure
    saveas(fig_vel, 'High_Eccentricity_Velocity.png');
end
\end{lstlisting}