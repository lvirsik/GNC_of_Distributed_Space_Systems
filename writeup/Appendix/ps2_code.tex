\subsection{Problem Set 2 Code}

\textbf{constants.m}
\begin{lstlisting}
classdef constants
    properties( Constant = true )
         mu = 3.986004 * 10^14 % m^3/s^2
         earth_radius = 6378000 % m
         J2 = 1.08263 * 10^(-3)
    end
 end
\end{lstlisting}

\textbf{dynamics.m}
\begin{lstlisting}
classdef dynamics
    methods (Static)
        function statedot = two_body_dynamics(t, state, simulation_settings)
            % Take in state vector and return statedot vector (accelerations)
            r = state(1:3);
            v = state(4:6);

            a = -constants.mu * r / (norm(r) ^ 3);

            if simulation_settings.J2
                a_j2_x = 1.5 * constants.J2 * constants.mu * (constants.earth_radius ^ 2 / norm(r) ^ 5) * (1 - (5 * (r(3) / norm(r)) ^ 2)) * r(1);
                a_j2_y = 1.5 * constants.J2 * constants.mu * (constants.earth_radius ^ 2 / norm(r) ^ 5) * (1 - (5 * (r(3) / norm(r)) ^ 2)) * r(2);
                a_j2_z = 1.5 * constants.J2 * constants.mu * (constants.earth_radius ^ 2 / norm(r) ^ 5) * (3 - (5 * (r(3) / norm(r)) ^ 2)) * r(3);
                a_j2 = [a_j2_x; a_j2_y; a_j2_z];
                a = a + a_j2;
            end

            r_dot = v;
            v_dot = a;

            statedot = [r_dot; v_dot];
        end

        function statedot = dynamics_with_relative(t, state, simulation_settings)
            % State input is relative state in rtn, followed by chief state in eci
            rho = state(1:3);
            drho = state(4:6);
            chief_state_rtn = util.ECI2RTN(state(7:12), state(7:12));
            r_0 = [norm(state(7:9)); 0; 0];
            v_0 = chief_state_rtn(4:6);

            h = norm(cross(r_0, v_0));
            omega = h / norm(r_0)^2;
            omega_dot = -2 * h * v_0(1) / (norm(r_0) ^ 3);
        
            k = -constants.mu / ((norm(r_0) + rho(1))^2 + rho(2)^2 + rho(3)^2)^(3/2);
            xddot = k * (norm(r_0) + rho(1)) + (constants.mu / norm(r_0)^2) + (2 * omega * drho(2)) + (omega_dot * rho(2)) + (rho(1) * omega^2);
            yddot = (k * rho(2)) - (2 * omega * drho(1)) - (omega_dot * rho(1)) + (rho(2) * omega^2);
            zddot = k * rho(3);

            ddrho = [xddot; yddot; zddot];

            chief_dot = dynamics.two_body_dynamics(t, state(7:12), simulation_settings);
            
            statedot = [drho; ddrho; chief_dot];
        end
    end
end
\end{lstlisting}

\textbf{sim\_config.m}
\begin{lstlisting}
clear all;
close all;
clc;

% Input initial conditions in orbital elments. Units of meters and degrees
initial_conditions_chief = [6780000, 0.0006, 51.6, 0, 0, 0]; % OEs
initial_conditions_chief = [6780000, 0, 0, 0, 0, 0]; 

% Time step and number of orbits
num_orbits = 3;
time_step = 1;

% Simulation settings
simulation_settings.numerical_propogation = true;
simulation_settings.keplerian_propogation = false;
simulation_settings.J2 = false;
simulation_settings.relative_deputy = true;
simulation_settings.absolute_deputy = true;

% Set graphics settings
graphics_settings.orbit_eci = false;
graphics_settings.compare_numerical_vs_kepler = false;
graphics_settings.plot_relative_position_deputy = true;
graphics_settings.plot_relative_position_deputy_comparison = true;
graphics_settings.plot_relative_position_error = true;
graphics_settings.plot_orbital_elements = struct();
graphics_settings.plot_orbital_elements.base_elems = false;
graphics_settings.plot_orbital_elements.J2_mean_analytical_solution = false;
graphics_settings.plot_eccentricity_vector = false;
graphics_settings.plot_ang_momentum_vector = false;
graphics_settings.plot_specific_energy = false;

% Case 1: Equal semi-major axis
disp('Running Case 1: Equal semi-major axis');
initial_conditions_deputy = [0; 0; 0; 1; 1; 0]; % Position and velocity relative to chief in RTN
sim1 = simulator(initial_conditions_chief, initial_conditions_deputy, num_orbits, time_step, simulation_settings, graphics_settings);
sim1.run_simulator();

% Case 2: Non-zero difference in semi-major axis
disp('Running Case 2: Non-zero difference in semi-major axis');
initial_conditions_deputy_case2 = [0; 0; 0; 10; 1; 0]; % Added 10 m/s in radial direction
sim2 = simulator(initial_conditions_chief, initial_conditions_deputy_case2, num_orbits, time_step, simulation_settings, graphics_settings);
sim2.run_simulator();

% Case 3: Calculate and apply maneuver to eliminate drift
disp('Running Case 3: Calculating and applying maneuver to bound motion');
simulation_settings.calculate_maneuver = true;
simulation_settings.apply_maneuver = true;
graphics_settings.plot_maneuver_comparison = true;
sim3 = simulator(initial_conditions_chief, initial_conditions_deputy_case2, num_orbits, time_step, simulation_settings, graphics_settings);
sim3.run_simulator();
\end{lstlisting}

\textbf{simulator.m}
\begin{lstlisting}
classdef simulator
    properties
        initial_conditions
        initial_conditions_deputy
        time_span
        dt
        simulation_settings
        graphics_settings
    end

    methods
        function self = simulator(initial_conditions_chief, initial_conditions_deputy, num_orbits, time_step, simulation_settings, graphics_settings)
            self.initial_conditions = initial_conditions_chief;
            self.initial_conditions_deputy = initial_conditions_deputy;
            time = (num_orbits * (2*pi*sqrt(self.initial_conditions(1)^3 / constants.mu)));
            time_span = 0:time_step:time;
            self.dt = time_step;
            self.time_span = time_span;
            self.simulation_settings = simulation_settings;
            self.graphics_settings = graphics_settings;
        end

        function run_simulator(self)
            a = self.initial_conditions(1);
            e = self.initial_conditions(2);
            i = deg2rad(self.initial_conditions(3));
            RAAN = deg2rad(self.initial_conditions(4));
            w = deg2rad(self.initial_conditions(5));
            v = deg2rad(self.initial_conditions(6));
            result.initial_conditions = self.initial_conditions;
            result.dt = self.dt;

            if self.simulation_settings.numerical_propogation
                initial_state_eci = util.OE2ECI(a, e, i, RAAN, w, v);

                % Perform Simulation
                options = odeset('RelTol', 1e-12, 'AbsTol', 1e-12);
                if self.simulation_settings.relative_deputy
                    initial_state = [self.initial_conditions_deputy; initial_state_eci];
                    [t, state_history] = ode45(@(t, state_history) dynamics.dynamics_with_relative(t, state_history, self.simulation_settings), self.time_span, initial_state, options);
                    result.state_history_num = state_history(:, 7:12);
                    result.relative_state_history = state_history(:, 1:6);
                    result.t_num = t;
                else
                    [t, state_history_num] = ode45(@(t, state_history_num) dynamics.two_body_dynamics(t, state_history_num, self.simulation_settings), self.time_span, initial_state_eci, options);
                    result.state_history_num = state_history_num;
                    result.t_num = t;
                end

                % Implementation for part (2c) - compute deputy orbit from chief orbit
                if self.simulation_settings.absolute_deputy
                    % Create initial state for deputy by applying variations to chief
                    % Convert relative state in RTN to ECI
                    rho_RTN = self.initial_conditions_deputy(1:3);
                    drho_RTN = self.initial_conditions_deputy(4:6);
                    
                    % Calculate RTN basis vectors directly
                    r_chief = initial_state_eci(1:3);
                    v_chief = initial_state_eci(4:6);
                    
                    R_hat = r_chief / norm(r_chief);
                    h_vec = cross(r_chief, v_chief);
                    N_hat = h_vec / norm(h_vec);
                    T_hat = cross(N_hat, R_hat);
                    
                    % RTN to ECI transformation matrix (each column is a basis vector)
                    R_rtn2eci = [R_hat, T_hat, N_hat];
                    
                    % Calculate deputy position in ECI
                    r_deputy_eci = r_chief + R_rtn2eci * rho_RTN;
                    
                    % Calculate angular velocity of RTN frame
                    omega = norm(h_vec) / (norm(r_chief)^2);
                    omega_vec = omega * N_hat;
                    
                    % Calculate deputy velocity in ECI
                    v_deputy_eci = v_chief + R_rtn2eci * drho_RTN + cross(omega_vec, R_rtn2eci * rho_RTN);
                    
                    % Combine to get deputy initial state in ECI
                    deputy_initial_state_eci = [r_deputy_eci; v_deputy_eci];
                    
                    % Propagate deputy orbit using the existing two_body_dynamics
                    [t_deputy, state_history_deputy] = ode45(@(t, state) dynamics.two_body_dynamics(t, state, self.simulation_settings), self.time_span, deputy_initial_state_eci, options);
                    
                    % Transform deputy orbit to RTN frame relative to chief
                    deputy_in_rtn = zeros(length(t), 6);
                    for j = 1:length(t)
                        % Get chief state at this time point
                        r_chief_j = result.state_history_num(j, 1:3)';
                        v_chief_j = result.state_history_num(j, 4:6)';
                        
                        % Calculate RTN basis vectors
                        R_hat_j = r_chief_j / norm(r_chief_j);
                        h_vec_j = cross(r_chief_j, v_chief_j);
                        N_hat_j = h_vec_j / norm(h_vec_j);
                        T_hat_j = cross(N_hat_j, R_hat_j);
                        
                        % ECI to RTN transformation matrix
                        R_eci2rtn_j = [R_hat_j, T_hat_j, N_hat_j]';
                        
                        % Deputy position and velocity in ECI
                        r_deputy_j = state_history_deputy(j, 1:3)';
                        v_deputy_j = state_history_deputy(j, 4:6)';
                        
                        % Calculate relative position in RTN
                        rho_j = R_eci2rtn_j * (r_deputy_j - r_chief_j);
                        
                        % Calculate relative velocity in RTN
                        omega_j = norm(h_vec_j) / (norm(r_chief_j)^2);
                        omega_vec_j = omega_j * N_hat_j;
                        drho_j = R_eci2rtn_j * (v_deputy_j - v_chief_j - cross(omega_vec_j, r_deputy_j - r_chief_j));
                        
                        deputy_in_rtn(j, :) = [rho_j; drho_j]';
                    end
                    
                    result.deputy_state_history = state_history_deputy;
                    result.deputy_in_rtn = deputy_in_rtn;
                    result.t_deputy = t_deputy;
                    
                    % Part (2e) and (2f) - Calculate and apply maneuver if requested
                    if isfield(self.simulation_settings, 'calculate_maneuver') && self.simulation_settings.calculate_maneuver
                        % Calculate the optimal maneuver to correct the drift
                        [delta_v, optimal_time_index, maneuver_point] = self.calculate_drift_correction(result.deputy_state_history, result.state_history_num, result.t_num);
                        
                        % Save maneuver information
                        result.maneuver_time = result.t_num(optimal_time_index);
                        result.maneuver_delta_v = delta_v;
                        result.maneuver_time_index = optimal_time_index;
                        result.maneuver_point = maneuver_point;
                        
                        % Apply maneuver if requested
                        if isfield(self.simulation_settings, 'apply_maneuver') && self.simulation_settings.apply_maneuver
                            % Get the state at maneuver point
                            maneuver_state = result.deputy_state_history(optimal_time_index, :);
                            
                            % Calculate the new velocity after maneuver
                            v_deputy = maneuver_state(4:6);
                            v_unit = v_deputy / norm(v_deputy);
                            
                            % We need to match chief's semi-major axis
                            a_chief = util.ECI2OE(result.state_history_num(optimal_time_index, :));
                            a_chief = a_chief(1);
                            r_deputy = norm(maneuver_state(1:3));
                            
                            % Calculate required velocity magnitude for the desired orbit
                            v_new_mag = sqrt(constants.mu * (2/r_deputy - 1/a_chief));
                            v_new = v_unit * v_new_mag;
                            
                            % Apply the maneuver post_maneuver_state = maneuver_state;
                            post_maneuver_state(4:6) = v_new;
                            
                            % Continue propagation from maneuver point
                            t_remaining = self.time_span(self.time_span > result.t_num(optimal_time_index));
                            if ~isempty(t_remaining)
                                [t_post, state_post] = ode45(@(t, state) dynamics.two_body_dynamics(t, state, self.simulation_settings), t_remaining, post_maneuver_state, options);
                                
                                % Store post-maneuver trajectory
                                t_combined = [result.t_num(1:optimal_time_index); t_post];
                                deputy_combined = [result.deputy_state_history(1:optimal_time_index, :); state_post];
                                
                                % Transform post-maneuver trajectory to RTN
                                deputy_rtn_post = zeros(length(t_post), 6);
                                for j = 1:length(t_post)
                                    % Find closest chief state time
                                    [~, t_idx] = min(abs(result.t_num - t_post(j)));
                                    
                                    % Chief state
                                    r_chief_j = result.state_history_num(t_idx, 1:3)';
                                    v_chief_j = result.state_history_num(t_idx, 4:6)';
                                    
                                    % Calculate RTN basis
                                    R_hat_j = r_chief_j / norm(r_chief_j);
                                    h_vec_j = cross(r_chief_j, v_chief_j);
                                    N_hat_j = h_vec_j / norm(h_vec_j);
                                    T_hat_j = cross(N_hat_j, R_hat_j);
                                    
                                    R_eci2rtn_j = [R_hat_j, T_hat_j, N_hat_j]';
                                    
                                    % Deputy state
                                    r_deputy_j = state_post(j, 1:3)';
                                    v_deputy_j = state_post(j, 4:6)';
                                    
                                    % Calculate relative position
                                    rho_j = R_eci2rtn_j * (r_deputy_j - r_chief_j);
                                    
                                    % Calculate relative velocity
                                    omega_j = norm(h_vec_j) / (norm(r_chief_j)^2);
                                    omega_vec_j = omega_j * N_hat_j;
                                    drho_j = R_eci2rtn_j * (v_deputy_j - v_chief_j - cross(omega_vec_j, r_deputy_j - r_chief_j));
                                    
                                    deputy_rtn_post(j, :) = [rho_j; drho_j]';
                                end
                                
                                % Combine trajectories
                                deputy_rtn_combined = [result.deputy_in_rtn(1:optimal_time_index, :); deputy_rtn_post];
                                
                                % Save combined results
                                result.t_combined = t_combined;
                                result.deputy_state_combined = deputy_combined;
                                result.deputy_in_rtn_combined = deputy_rtn_combined;
                            end
                        end
                    end
                end

                % Gather orbital element history and other interesting information
                oe_history = zeros(length(result.state_history_num), 6);
                ecc_vector_history = zeros(length(result.state_history_num), 3);
                ang_mom_history = zeros(length(result.state_history_num), 3);
                energy_history = zeros(length(result.state_history_num), 1);
                for j = 1:length(result.state_history_num)
                    oe_history(j, :) = util.ECI2OE(result.state_history_num(j, :));
                    ecc_vector_history(j, :) = util.get_ecc_vector(result.state_history_num(j, :));
                    ang_mom_history(j, :) = util.get_ang_momentum(result.state_history_num(j, :));
                    energy_history(j, :) = util.get_energy(result.state_history_num(j, :));
                end
                result.oe_history = oe_history;
                result.ecc_vector_history = ecc_vector_history;
                result.ang_mom_history = ang_mom_history;
                result.energy_history = energy_history;
            end

            if self.simulation_settings.keplerian_propogation
                n = sqrt(constants.mu / a ^ 3);
                state_history_kep = zeros(length(self.time_span), 6);
                for j=1:length(self.time_span)
                    state_kep = util.OE2ECI(a, e, i, RAAN, w, v);
                    state_history_kep(j,:) = state_kep';
                    E = 2 * atan2(tan(v/2) * sqrt((1 - e) / (1 + e)), 1);
                    M = E - e*sin(E);
                    M_new = M + (n * self.dt);
                    E_new = util.MtoE(M_new, e, 10^(-9));
                    v_new = 2 * atan2(sqrt((1 + e) / (1 - e)) * tan(E_new / 2), 1);
                    v = v_new;
                end
                result.state_history_kep = state_history_kep;
                result.t_kep = self.time_span;
            end
            plotter(result, self.graphics_settings);
        end
        
        function [delta_v, optimal_time_index, maneuver_point] = calculate_drift_correction(self, deputy_state_history, chief_state_history, t)
            % Calculate orbital elements for both satellites
            deputy_oe = zeros(length(t), 6);
            chief_oe = zeros(length(t), 6);
            
            for i = 1:length(t)
                deputy_oe(i, :) = util.ECI2OE(deputy_state_history(i, :));
                chief_oe(i, :) = util.ECI2OE(chief_state_history(i, :));
            end
            
            % Calculate semi-major axis difference
            delta_a = deputy_oe(:, 1) - chief_oe(:, 1);
            disp(['Current semi-major axis difference: ', num2str(delta_a(end)), ' meters']);
            
            % Find points of minimum and maximum radius (approximate apogee/perigee)
            deputy_radius = zeros(length(t), 1);
            for i = 1:length(t)
                deputy_radius(i) = norm(deputy_state_history(i, 1:3));
            end
            
            [~, min_r_idx] = min(deputy_radius);
            [~, max_r_idx] = max(deputy_radius);
            
            % Calculate velocities at these points
            v_min_r = norm(deputy_state_history(min_r_idx, 4:6));
            v_max_r = norm(deputy_state_history(max_r_idx, 4:6));
            
            % Calculate required velocities for bounded motion
            a_target = chief_oe(1, 1); % Target semi-major axis = chief's
            r_min = deputy_radius(min_r_idx);
            r_max = deputy_radius(max_r_idx);
            
            v_req_min_r = sqrt(constants.mu * (2/r_min - 1/a_target));
            v_req_max_r = sqrt(constants.mu * (2/r_max - 1/a_target));
            
            % Calculate delta-v at both locations
            dv_min_r = abs(v_req_min_r - v_min_r);
            dv_max_r = abs(v_req_max_r - v_max_r);
            
            % Choose the more efficient maneuver
            if dv_min_r <= dv_max_r
                delta_v = dv_min_r;
                optimal_time_index = min_r_idx;
                maneuver_point = [r_min, t(min_r_idx)];
                disp(['Optimal maneuver at minimum radius (perigee), delta-v = ', num2str(delta_v), ' m/s']);
            else
                delta_v = dv_max_r;
                optimal_time_index = max_r_idx;
                maneuver_point = [r_max, t(max_r_idx)];
                disp(['Optimal maneuver at maximum radius (apogee), delta-v = ', num2str(delta_v), ' m/s']);
            end
        end
    end
end
\end{lstlisting}

\textbf{util.m}
\begin{lstlisting}
classdef util
    methods (Static)
        function state_eci = OE2ECI(a, e, i, RAAN, w, v)
            p = a * (1 - e^2);
            r = p / (1 + e*cos(v));
            rPQW = [r*cos(v); r*sin(v); 0];
            vPQW = [sqrt(constants.mu/p) * -sin(v); sqrt(constants.mu/p)*(e + cos(v)); 0];
        
            R1 = [cos(-RAAN) sin(-RAAN) 0;...
                -sin(-RAAN) cos(-RAAN) 0;...
                0       0       1];
            R2 = [1  0        0;...
                0  cos(-i) sin(-i);...
                0 -sin(-i) cos(-i)];
            R3 = [cos(-w) sin(-w) 0;...
                -sin(-w) cos(-w) 0;...
                0        0        1];
            R = R1 * R2 * R3;
        
            rECI = R * rPQW;
            vECI = R * vPQW;
            state_eci = [rECI; vECI];
        end

        function H = get_ang_momentum(state)
            R = state(1:3);
            V = state(4:6);
            H = cross(R, V);
        end

        function E = get_ecc_vector(state)
            R = state(1:3);
            V = state(4:6);
            H = util.get_ang_momentum(state);
            mu = constants.mu;
            E = (cross(V,H)/mu)-(R/norm(R));
        end

        function E = get_energy(state)
            oes = util.ECI2OE(state);
            a = oes(1);
            E = -constants.mu / (2 * a);
        end

        function OEs = ECI2OE(state)
            mu = constants.mu;
            R = state(1:3);
            V = state(4:6);
            r = norm(R);
            
            %Get to Perifocal
            H = util.get_ang_momentum(state);
            h = norm(H);
            E = util.get_ecc_vector(state);
            e = norm(E);
            p = (h)^2/mu;
            a = p / (1 - e^2);
            n = sqrt(mu/a^3);
            E = atan2((dot(R, V)/(n*a^2)), (1-(r/a)));
            v = 2 * atan(sqrt((1+e)/(1-e)) * tan(E/2));
            
            %Find Angles
            W = [H(1)/h H(2)/h H(3)/h];
            i = atan2(sqrt(W(1)^2 + W(2)^2), W(3));
            RAAN = atan2(W(1), -W(2));
            w = atan2((R(3)/sin(i)), R(1)*cos(RAAN)+R(2)*sin(RAAN)) - v;
            if rad2deg(w) > 180
                w = w - 2*pi;
            end
            
            OEs = [a, e, i, RAAN, w, v];
        end

        function state_pqw = OE2PQW(a, e, i, RAAN, w, v)
            p = a * (1 - e^2);
            r = p / (1 + e*cos*(v));
            P = r*cos(v);
            Q = r*sin(v);
            W = 0;
            rPQW = [P; Q; W];
            vPQW = sqrt(constants.mu / p) * [-sin(v); e + cos(v); 0];
            state_pqw = [rPQW; vPQW];
        end

        function E = MtoE(M, e, err)
            E = M; N = 0;
            delta = (E - e*sin(E) - M) / (1 - e*cos(E));
            while (delta > err || N < 100)
                delta = (E - e*sin(E) - M) / (1 - e*cos(E));
                E = E - delta;
                N = N + 1;
            end
        end

        function state_rtn = ECI2RTN(state_eci, reference_state_eci)
            r_eci = reference_state_eci(1:3);
            v_eci = reference_state_eci(4:6);
            n = cross(r_eci, v_eci);

            R = r_eci / norm(r_eci);
            N = n / norm(n);
            T = cross(N, R);

            R_eci2rtn = inv([R, T, N]);
            diff = state_eci(1:3) - reference_state_eci(1:3);

            r_rtn = R_eci2rtn * diff;
            
            f_dot = norm(cross(r_eci, v_eci)) / norm(r_eci)^2;
            w = [0,0,f_dot]';
            v_rtn =  R_eci2rtn * state_eci(4:6) - cross(w, r_rtn);
            state_rtn = [r_rtn; v_rtn];
        end

        function R_eci2rtn = R_ECI2RTN(reference_state_eci)
            r_eci = reference_state_eci(1:3)';
            v_eci = reference_state_eci(4:6)';
            n = cross(r_eci, v_eci);

            R = r_eci / norm(r_eci);
            N = n / norm(n);
            T = cross(N, R);

            R_eci2rtn = [R, T, N];
        end
    end
end
\end{lstlisting}

\textbf{plotter.m}
\begin{lstlisting}
function plotter(result, graphics_settings)
    % Create figures directory in the current working directory
    figures_dir = 'figures';
    
    % Get the case identifier
    if isfield(result , 'maneuver_delta_v')
        case_str = 'Case3_';
    elseif isfield(result, 'deputy_in_rtn') && max(abs(result.deputy_in_rtn(:,1))) > 5000
        case_str = 'Case2_';
    else
        case_str = 'Case1_';
    end
    
    % Counter for figures
    fig_count = 0;
    
    % Close any existing figures to avoid interference
    close all;
    
    % Display all figures and save them
    if graphics_settings.orbit_eci
        fig_count = fig_count + 1;
        fig = figure('Name', 'OrbitECI');
        plot_orbit_eci(result);
        saveas(fig, ['./figures/', case_str, 'OrbitECI.png']);
    end

    if graphics_settings.compare_numerical_vs_kepler
        fig_count = fig_count + 1;
        fig = figure('Name', 'NumVsKep');
        compare_numerical_vs_kepler(result);
        saveas(fig, ['./figures/', case_str, 'NumVsKep.png']);
    end

    if graphics_settings.plot_relative_position_deputy
        % Position figure
        fig_count = fig_count + 1;
        fig_traj = figure('Name', 'RelativeTrajectory');
        
        rho = result.relative_state_history(:, 1:6);
        R = rho(:,1); T = rho(:,2); N = rho(:,3);
        
        hold on;
        plot3(T, N, R, 'b', 'LineWidth', 2);
        xlabel('Tangential (m)');
        ylabel('Normal (m)');
        zlabel('Radial (m)');
        title('Deputy Trajectory in RTN Frame');
        grid on;
        view(3);
        
        saveas(fig_traj, ['./figures/', case_str, 'RelativeTrajectory.png']);
        
        % Velocity figure
        fig_count = fig_count + 1;
        fig_vel = figure('Name', 'RelativeVelocity');
        
        Rv = rho(:,4); Tv = rho(:,5); Nv = rho(:,6);
        
        hold on;
        plot3(Tv*100, Nv*100, Rv*100, 'r', 'LineWidth', 2);
        xlabel('Tangential (cm/s)');
        ylabel('Normal (cm/s)');
        zlabel('Radial (cm/s)');
        title('Deputy Velocity in RTN Frame');
        grid on;
        view(3);
        
        saveas(fig_vel, ['./figures/', case_str, 'RelativeVelocity.png']);
    end
    
    if isfield(graphics_settings, 'plot_relative_position_deputy_comparison') && graphics_settings.plot_relative_position_deputy_comparison && isfield(result, 'deputy_in_rtn')
        fig_count = fig_count + 1;
        fig = figure('Name', 'MethodsComparison');
        
        % For (2b) - relative motion from nonlinear equations
        rho = result.relative_state_history(:, 1:3);
        
        % For (2c) - relative motion computed from differencing individual orbits
        rho_diff = result.deputy_in_rtn(:, 1:3);
        
        hold on;
        plot3(rho(:,2), rho(:,3), rho(:,1), 'b', 'LineWidth', 2);
        plot3(rho_diff(:,2), rho_diff(:,3), rho_diff(:,1), 'r--', 'LineWidth', 1.5);
        xlabel('Tangential (m)');
        ylabel('Normal (m)');
        zlabel('Radial (m)');
        title('Deputy Trajectory in RTN Frame - Methods Comparison');
        legend('Nonlinear Relative Equations', 'Differencing Individual Orbits');
        grid on;
        view(3);
        
        saveas(fig, ['./figures/', case_str, 'MethodsComparison.png']);
    end
    
    if isfield(graphics_settings, 'plot_relative_position_error') && graphics_settings.plot_relative_position_error && isfield(result, 'deputy_in_rtn')
        fig_count = fig_count + 1;
        fig_error = figure('Name', 'MethodsError');
        
        % Calculate error between nonlinear equations and differencing orbits
        rho_nonlinear = result.relative_state_history(:, 1:3);
        rho_differencing = result.deputy_in_rtn(:, 1:3);
        
        % Ensure they're the same length for comparison
        min_length = min(size(rho_nonlinear, 1), size(rho_differencing, 1));
        rho_nonlinear = rho_nonlinear(1:min_length, :);
        rho_differencing = rho_differencing(1:min_length, :);
        
        % Calculate error in each component
        error = rho_nonlinear - rho_differencing;
        
        % Calculate absolute error
        abs_error = sqrt(sum(error.^2, 2));
        
        % Plot errors - using the current figure (fig_error)
        subplot(2,1,1);
        hold on;
        plot(result.t_num(1:min_length) / (60*60), error(:,1), 'r');
        plot(result.t_num(1:min_length) / (60*60), error(:,2), 'g');
        plot(result.t_num(1:min_length) / (60*60), error(:,3), 'b');
        grid on;
        xlabel('Time (hrs)');
        ylabel('Error (m)');
        title('Component Error Between Nonlinear Equations and Differencing Orbits');
        legend('Radial Error', 'Tangential Error', 'Normal Error');
        
        subplot(2,1,2);
        plot(result.t_num(1:min_length) / (60*60), abs_error, 'k');
        grid on;
        xlabel('Time (hrs)');
        ylabel('Absolute Error (m)');
        title('Absolute Error Between Methods');
        
        % Display statistics
        max_error = max(abs_error);
        mean_error = mean(abs_error);
        disp(['Maximum absolute error: ', num2str(max_error), ' meters']);
        disp(['Mean absolute error: ', num2str(mean_error), ' meters']);
        
        saveas(fig_error, ['./figures/', case_str, 'MethodsError.png']);
    end
    
    if isfield(graphics_settings, 'plot_maneuver_comparison') && graphics_settings.plot_maneuver_comparison && isfield(result, 'deputy_in_rtn_combined')
        fig_count = fig_count + 1;
        fig = figure('Name', 'ManeuverComparison');
        
        % Check that the necessary fields exist
        if ~isfield(result, 'deputy_in_rtn_combined')
            disp('No maneuver applied or combined trajectory not available');
        else
            % Original trajectory (pre-maneuver)
            rho_original = result.deputy_in_rtn(:, 1:3);
            
            % Combined trajectory (with maneuver)
            rho_combined = result.deputy_in_rtn_combined(:, 1:3);
            
            % Find maneuver point
            maneuver_idx = result.maneuver_time_index;
            maneuver_point = rho_original(maneuver_idx, :);
            
            hold on;
            plot3(rho_original(:,2), rho_original(:,3), rho_original(:,1), 'r', 'LineWidth', 1.5);
            plot3(rho_combined(:,2), rho_combined(:,3), rho_combined(:,1), 'b', 'LineWidth', 1.5);
            plot3(maneuver_point(2), maneuver_point(3), maneuver_point(1), 'go', 'MarkerSize', 8, 'MarkerFaceColor', 'g');
            
            xlabel('Tangential (m)');
            ylabel('Normal (m)');
            zlabel('Radial (m)');
            title('Deputy Trajectory Before and After Maneuver');
            legend('Original Trajectory (Unbounded)', 'Post-Maneuver Trajectory (Bounded)', 'Maneuver Location');
            grid on;
            view(3);
            
            % Display maneuver information
            annotation('textbox', [0.15, 0.05, 0.7, 0.1], 'String', ...
                {['Maneuver at t = ', num2str(result.maneuver_time/3600, '%.2f'), ' hours'], ...
                 ['Delta-v magnitude = ', num2str(result.maneuver_delta_v, '%.4f'), ' m/s']}, ...
                'FitBoxToText', 'on', 'BackgroundColor', 'white');
        end
        
        saveas(fig, ['./figures/', case_str, 'ManeuverComparison.png']);
    end
    
    disp(['Generated ' num2str(fig_count) ' figures and saved them to ./figures/ directory']);
end
\end{lstlisting}