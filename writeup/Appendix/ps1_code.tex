\subsection{Problem Set 1 Code}

\textbf{constants.m}
\begin{lstlisting}
classdef constants
    properties( Constant = true )
         mu = 3.986004 * 10^14 % m^3/s^2
         earth_radius = 6378000 % m
         J2 = 1.08263 * 10^(-3)
    end
 end
\end{lstlisting}

\textbf{dynamics.m}
\begin{lstlisting}
function statedot = dynamics(t, state, simulation_settings)
% Take in state vector and return statedot vector (accelerations)
r = state(1:3);
v = state(4:6);

a = -constants.mu * r / (norm(r) ^ 3);

if simulation_settings.J2
    a_j2_x = 1.5 * constants.J2 * constants.mu * (constants.earth_radius ^ 2 / norm(r) ^ 5) * (1 - (5 * (r(3) / norm(r)) ^ 2)) * r(1);
    a_j2_y = 1.5 * constants.J2 * constants.mu * (constants.earth_radius ^ 2 / norm(r) ^ 5) * (1 - (5 * (r(3) / norm(r)) ^ 2)) * r(2);
    a_j2_z = 1.5 * constants.J2 * constants.mu * (constants.earth_radius ^ 2 / norm(r) ^ 5) * (3 - (5 * (r(3) / norm(r)) ^ 2)) * r(3);
    a_j2 = [a_j2_x; a_j2_y; a_j2_z];
    a = a + a_j2;
end

r_dot = v;
v_dot = a;

statedot = [r_dot; v_dot];

end
\end{lstlisting}

\textbf{plotter.m}
\begin{lstlisting}
function plotter(result, graphics_settings)
    if graphics_settings.orbit_eci
        plot_orbit_eci(result)
    end

    if graphics_settings.compare_numerical_vs_kepler
        compare_numerical_vs_kepler(result)
    end

    if graphics_settings.plot_orbital_elements.base_elems
        plot_orbital_elements(result, graphics_settings)
    end

    if graphics_settings.plot_eccentricity_vector
        plot_eccentricity_vector(result)
    end

    if graphics_settings.plot_ang_momentum_vector
        plot_ang_momentum_vector(result)
    end

    if graphics_settings.plot_specific_energy
        plot_specific_energy(result)
    end
end

function compare_numerical_vs_kepler(result)
    t = result.t_num;
    state_history_num = result.state_history_num;
    t_span = result.t_kep;
    state_history_kep = result.state_history_kep;

    state_history_kep = interp1(t_span, state_history_kep, t, 'linear');

    error_rtn = zeros(size(state_history_kep));
    for j = 1:size(state_history_kep, 1)
        error_rtn(j, :) = util.ECI2RTN(state_history_kep(j, :)) - util.ECI2RTN(state_history_num(j, :));
    end
    
    figure;
    hold on;
    plot(t, error_rtn)
    xlabel('Time (s)');
    ylabel('Absolute Error (m and m/s)');
    title('Absolute Error vs Time, RTN Frame');
    legend('Error R_r', 'Error R_t', 'Error R_n', 'Error V_r', 'Error V_t', 'Error V_n');

end

function plot_orbit_eci(result)
    state_history = result.state_history_num;

    % Extract positions
    x = state_history(:, 1);
    y = state_history(:, 2);
    z = state_history(:, 3);

    % Show Earth
    [xe, ye, ze] = sphere(50);
    xe = constants.earth_radius * xe;
    ye = constants.earth_radius * ye;
    ze = constants.earth_radius * ze;

    figure;

    % Plot Earth
    surf(xe, ye, ze, 'FaceColor', 'blue', 'EdgeColor', 'none');
    hold on;

    % Plot Orbit
    plot3(x, y, z, 'c-', 'LineWidth', 2); 
    grid on;
    xlabel('X');
    ylabel('Y');
    zlabel('Z');
    title('ECI Trajectory');
    view(3);
    axis equal;
end

function plot_orbital_elements(result, graphics_settings)
    t = result.t_num;
    oe_history = result.oe_history;

    a     = oe_history(:, 1);
    e     = oe_history(:, 2);
    i     = oe_history(:, 3);
    RAAN  = oe_history(:, 4);  
    omega = oe_history(:, 5);
    nu    = oe_history(:, 6);  

    if graphics_settings.plot_orbital_elements.J2_mean_analytical_solution
        J2_analytical = zeros(length(result.state_history_num), 2);
        for j = 1:length(result.state_history_num)
            K = -(3 * constants.J2 * sqrt(constants.mu) * constants.earth_radius^2) / (2 * ((1 - e(j)^2)^2) * (a(j) ^ (7/2)));
            RAAN_j2 = result.initial_conditions(4) - (K * t(j) * cos(i(j)));
            w_j2 = result.initial_conditions(5) + ((K / 2) * t(j) * ((5 * (cos(i(j)) ^ 2)) - 1));
            J2_analytical(j, :) = [RAAN_j2, w_j2];
        end
    end

    figure;
    tiledlayout(3,2, 'Padding', 'compact', 'TileSpacing', 'compact');

    nexttile;
    plot(t, a, 'r'); grid on;
    xlabel('Time'); ylabel('a (km)');
    title('Semi-Major Axis');

    nexttile;
    plot(t, e, 'g'); grid on;
    xlabel('Time'); ylabel('e');
    title('Eccentricity');

    nexttile;
    plot(t, rad2deg(i), 'y'); grid on;
    xlabel('Time'); ylabel('Inclination (deg)');
    title('Inclination');

    nexttile;
    hold on;
    plot(t, rad2deg(RAAN), 'm'); grid on;
    if graphics_settings.plot_orbital_elements.J2_mean_analytical_solution
        plot(t, rad2deg(J2_analytical(:,1)), 'k');
        legend('Numerical', 'J2 Mean Analytical');
    end
    xlabel('Time'); ylabel('RAAN (deg)');
    title('RAAN');

    nexttile;
    hold on;
    plot(t, rad2deg(omega), 'c'); grid on;
    if graphics_settings.plot_orbital_elements.J2_mean_analytical_solution
        plot(t, rad2deg(J2_analytical(:,2)), 'k');
        legend('Numerical', 'J2 Mean Analytical');
    end
    xlabel('Time'); ylabel('\omega (deg)');
    title('Argument of Periapsis');
    

    nexttile;
    plot(t, rad2deg(nu), 'k'); grid on;
    xlabel('Time'); ylabel('\nu (deg)');
    title('True Anomaly');
end

function plot_eccentricity_vector(result)
    figure;
    hold on
    plot(result.t_num, result.ecc_vector_history(:,1), 'r'); 
    plot(result.t_num, result.ecc_vector_history(:,2), 'g');
    plot(result.t_num, result.ecc_vector_history(:,3), 'b');
    grid on;
    xlabel('Time (s)'); ylabel('Eccentricity Vector');
    title('Eccentricity Vector over Time');
    legend('X_comp', 'Y_comp', 'Z_comp');
end

function plot_ang_momentum_vector(result)
    figure;
    hold on
    plot(result.t_num, result.ang_mom_history(:,1), 'r'); 
    plot(result.t_num, result.ang_mom_history(:,2), 'g');
    plot(result.t_num, result.ang_mom_history(:,3), 'b');
    grid on;
    xlabel('Time (s)'); ylabel('Angular Momentum Vector');
    title('Angular Momentum Vector over Time');
    legend('X_comp', 'Y_comp', 'Z_comp');
end

function plot_specific_energy(result)
    figure;
    hold on
    plot(result.t_num, result.energy_history, 'b'); 
    grid on;
    xlabel('Time (s)'); ylabel('Specific Energy (J)');
    title('Specific Energy over Time');
end
\end{lstlisting}

\textbf{sim\_config.m}
\begin{lstlisting}
% Input initial conditions in orbital elments. Units of meters and degrees
initial_conditions = [10000000, 0.1, 45, 0, 0, 0]; % OEs

% Input time span of simulation in start:time_step:end format. Units of seconds
time_span = 0:1:500000;

% Simulation settings
simulation_settings.numerical_propogation = true;
simulation_settings.keplerian_propogation = false;
simulation_settings.J2 = true;

% Set graphics settings for graph output
graphics_settings.orbit_eci = true;
graphics_settings.compare_numerical_vs_kepler = false;

graphics_settings.plot_orbital_elements = struct();
graphics_settings.plot_orbital_elements.base_elems = true;
graphics_settings.plot_orbital_elements.J2_mean_analytical_solution = false;

graphics_settings.plot_eccentricity_vector = false;
graphics_settings.plot_ang_momentum_vector = false;
graphics_settings.plot_specific_energy = false;


sim = simulator(initial_conditions, time_span, simulation_settings, graphics_settings);
sim.run_simulator();
\end{lstlisting}

\textbf{simulator.m}
\begin{lstlisting}
classdef simulator
    properties
        initial_conditions
        time_span
        simulation_settings
        graphics_settings
    end

    methods
        function self = simulator(initial_conditions, time_span, simulation_settings, graphics_settings)
            self.initial_conditions = initial_conditions;
            self.time_span = time_span;
            self.simulation_settings = simulation_settings;
            self.graphics_settings = graphics_settings;
        end

        function run_simulator(self)
            a = self.initial_conditions(1);
            e = self.initial_conditions(2);
            i = deg2rad(self.initial_conditions(3));
            RAAN = deg2rad(self.initial_conditions(4));
            w = deg2rad(self.initial_conditions(5));
            v = deg2rad(self.initial_conditions(6));
            result.initial_conditions = self.initial_conditions;

            if self.simulation_settings.numerical_propogation
                initial_state = util.OE2ECI(a, e, i, RAAN, w, v);

                % Perform Simulation
                options = odeset('RelTol', 1e-7, 'AbsTol', 1e-9);
                [t, state_history_num] = ode45(@(t, state_history_num) dynamics(t, state_history_num, self.simulation_settings), [0 self.time_span(end)], initial_state, options);
                result.state_history_num = state_history_num;
                result.t_num = t;

                % Gather orbital element history and other interesting information
                oe_history = zeros(length(state_history_num), 6);
                ecc_vector_history = zeros(length(state_history_num), 3);
                ang_mom_history = zeros(length(state_history_num), 3);
                energy_history = zeros(length(state_history_num), 1);
                for j = 1:length(state_history_num)
                    oe_history(j, :) = util.ECI2OE(state_history_num(j, :));
                    ecc_vector_history(j, :) = util.get_ecc_vector(state_history_num(j, :));
                    ang_mom_history(j, :) = util.get_ang_momentum(state_history_num(j, :));
                    energy_history(j, :) = util.get_energy(state_history_num(j, :));
                end
                result.oe_history = oe_history;
                result.ecc_vector_history = ecc_vector_history;
                result.ang_mom_history = ang_mom_history;
                result.energy_history = energy_history;
            end

            if self.simulation_settings.keplerian_propogation
                n = sqrt(constants.mu / a ^ 3);
                dt = self.time_span(2) - self.time_span(1);
                state_history_kep = zeros(length(self.time_span), 6);
                for j=1:length(self.time_span)
                    state_kep = util.OE2ECI(a, e, i, RAAN, w, v);
                    state_history_kep(j,:) = state_kep';
                    E = 2 * atan2(tan(v/2) * sqrt((1 - e) / (1 + e)), 1);
                    M = E - e*sin(E);
                    M_new = M + (n * dt);
                    E_new = util.MtoE(M_new, e, 10^(-6));
                    v_new = 2 * atan2(sqrt((1 + e) / (1 - e)) * tan(E_new / 2), 1);
                    v = v_new;
                end
                result.state_history_kep = state_history_kep;
                result.t_kep = self.time_span;
            end
            plotter(result, self.graphics_settings)
        end
    end
end
\end{lstlisting}

\textbf{util.m}
\begin{lstlisting}
classdef util
    methods (Static)
        function state_eci = OE2ECI(a, e, i, RAAN, w, v)
            p = a * (1 - e^2);
            r = p / (1 + e*cos(v));
            rPQW = [r*cos(v); r*sin(v); 0];
            vPQW = [sqrt(constants.mu/p) * -sin(v); sqrt(constants.mu/p)*(e + cos(v)); 0];
        
            R1 = [cos(-RAAN) sin(-RAAN) 0;...
                -sin(-RAAN) cos(-RAAN) 0;...
                0       0       1];
            R2 = [1  0        0;...
                0  cos(-i) sin(-i);...
                0 -sin(-i) cos(-i)];
            R3 = [cos(-w) sin(-w) 0;...
                -sin(-w) cos(-w) 0;...
                0        0        1]; R = R1 * R2 * R3;
        
            rECI = R * rPQW;
            vECI = R * vPQW;
            state_eci = [rECI; vECI];
        end

        function H = get_ang_momentum(state)
            R = state(1:3);
            V = state(4:6);
            H = cross(R, V);
        end

        function E = get_ecc_vector(state)
            R = state(1:3);
            V = state(4:6);
            H = util.get_ang_momentum(state);
            mu = constants.mu;
            E = (cross(V,H)/mu)-(R/norm(R));
        end

        function E = get_energy(state)
            oes = util.ECI2OE(state);
            a = oes(1);
            E = -constants.mu / (2 * a);
        end

        function OEs = ECI2OE(state)
            mu = constants.mu;
            R = state(1:3);
            V = state(4:6);
            r = norm(R);
            
            %Get to Perifocal
            H = util.get_ang_momentum(state);
            h = norm(H);
            E = util.get_ecc_vector(state);
            e = norm(E);
            p = (h)^2/mu;
            a = p / (1 - e^2);
            n = sqrt(mu/a^3);
            E = atan2((dot(R, V)/(n*a^2)), (1-(r/a)));
            v = 2 * atan(sqrt((1+e)/(1-e)) * tan(E/2));
            
            %Find Angles
            W = [H(1)/h H(2)/h H(3)/h];
            i = atan2(sqrt(W(1)^2 + W(2)^2), W(3));
            RAAN = atan2(W(1), -W(2));
            w = atan2((R(3)/sin(i)), R(1)*cos(RAAN)+R(2)*sin(RAAN)) - v;
            if rad2deg(w) > 180
                w = w - 2*pi;
            end
            
            OEs = [a, e, i, RAAN, w, v];
        end

        function state_pqw = OE2PQW(a, e, i, RAAN, w, v)
            p = a * (1 - e^2);
            r = p / (1 + e*cos*(v));
            P = r*cos(v);
            Q = r*sin(v);
            W = 0;
            rPQW = [P; Q; W];
            vPQW = sqrt(constants.mu / p) * [-sin(v); e + cos(v); 0];
            state_pqw = [rPQW; vPQW];
        end

        function E = MtoE(M, e, err)
            E = M; N = 0;
            delta = (E - e*sin(E) - M) / (1 - e*cos(E));
            while (delta > err || N < 100)
                delta = (E - e*sin(E) - M) / (1 - e*cos(E));
                E = E - delta;
                N = N + 1;
            end
        end

        function state_rtn = ECI2RTN(state_eci)
            r_eci = state_eci(1:3)';
            v_eci = state_eci(4:6)';
            n = cross(r_eci, v_eci);

            R = r_eci / norm(r_eci);
            N = n / norm(n);
            T = cross(N, R);

            R_eci2rtn = [R, T, N];

            r_rtn = R_eci2rtn * r_eci;

            f_dot = norm(cross(r_eci, v_eci)) / norm(r_eci)^2;
            w = [0,0,f_dot]';
            v_rtn = (R_eci2rtn * v_eci) - cross(w, r_rtn);

            state_rtn = [r_rtn; v_rtn];
        end
    end
 end
\end{lstlisting}